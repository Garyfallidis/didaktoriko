\documentclass{bioinfo}
\usepackage{url}

\usepackage[british,english]{babel}
\usepackage{mathpazo}
\usepackage[T1]{fontenc}
% \usepackage[latin9]{inputenc}
\usepackage{float}
\usepackage{bm}
\usepackage{nicefrac}
\usepackage{amsmath}
\usepackage{graphicx}
\usepackage{setspace}
\usepackage{amssymb}
\usepackage{natbib}
\usepackage[title]{appendix}
\usepackage{siunitx}
\usepackage{chngcntr}
\usepackage{algorithmic}
\renewcommand{\algorithmicrequire}{\textbf{Input:}}
\renewcommand{\algorithmicensure}{\textbf{Output:}}

\makeatletter
\newfloat{algorithm}{H}{loa}[section]
\floatname{algorithm}{Algorithm}
\counterwithout{algorithm}{algorithm}
\def\argmin{\mathop{\operator@font arg\,min}} 
\def\argmax{\mathop{\operator@font arg\,max}} 
\makeatother

\copyrightyear{}
\pubyear{}

\begin{document}
\firstpage{1}

\title[EIT]{Diffusion Imaging Q-Space Reconstructions on Cartesian Lattices}

\author[Garyfallidis, Yeh and
Nimmo-Smith]{Eleftherios~Garyfallidis\,$^{1,4,*}$, Frank~Yeh\,$^{2,*}$, Maxime~Descoteaux\,$^{3,*}$, and
  Ian~Nimmo-Smith\,$^{4}$\footnote{to whom correspondence should be
    addressed. e-mail: garyfallidis@gmail.com}}

\address{\,$^{1}$Wolfson College, University of Cambridge, Cambridge, UK\\
  \,$^{2}$Carnegie~Mellon, Pittsburgh.\\
  \,$^{3}$University of Sherbrooke, Canada.\\
  \,$^{4}$MRC Cognition and Brain Sciences Unit, Cambridge, UK.\\}

\history{}

\editor{}

\maketitle

\begin{abstract}
\noindent
u
\section{Keywords:} Diffusion Spectrum Imaging, Q-Ball Imaging,
Diffusion MRI.

\end{abstract}

\section{Introduction}

We use the term \emph{reconstruction} to mean the identification
(estimation) of fiber directions in a voxel of diffusion weighted MR
data. We will identify some principles describing the geometric and
physical proerties of the dMR signal to motive a new broad class of
reconstruciotn methods. Distinction between parametric (tensor,
multitensor, sticks-and-ball, kurtosis) and non-parametric methods (DSI,
GQI, spherical harmonic models). For non-parametric methods typically
derive a function $\psi$ on a 2-sphere $S^2$. This sphere represents all
possible fiber orientations. Typically the larger isolated maxima of
$\psi$ on $S^2$ are taken as local estimates of directions of underlying
fibers.

Fiber = real fiber, a packet of threads

Tract = fiber in the central nervous system (whose threads are axons)

ODF = orientation density function (not necessarily normalised). However
some ODFs are based on probabilistic assumptions where normalisation is
required - in this case ODF stands for orientation distribution function.

reconstruction sphere = spherical grid (mesh) whose vertices are the
locations where the ODF is evaluated.

estimated fiber directions = peaks (local maxima) of the ODF on the
reconstruction sphere

Goal: (1) a simple and general way to understand the diffusion
signal. (2) reliable resolution of low angle fiber crossings. (3)
consideration to performance (speed) of software implementations.
Principles:

we work directly on the measured signal in q-space -- dMR gradient
spaces -- (no Fourier transform required)

uses spatio-geometric properties of the signal - orthogonal information
+ additivity + decay

FRT works only on spherical shells - inner and outer evaluations of the
signal have something to contribute.

Introduce a framework $\int_r \int_{orth} F O$ which creates an ODF
where the equatorial radial projection of the contribution of signal is
based on $F$-transformed signal strength, and $O$-transformed radial
distance weighting

Limitation: data needs to be gathered on a suitably dense cartesian grid
in q-space - but can see how to extend to multiple shell acquisition
schemes.


\subsection{Overview}

\subsection{Theory}

\begin{methods}

\section{Methods and Materials}

\subsection{Equatorial Inversion Transform}

\subsection{Other methods}

\subsection{Implementation }

\subsubsection{Standard EIT\label{sub:Standard-EIT}}

\subsubsection{Fast EIT}

\subsection{Peak Finding\label{sub:Peak-Finding}}

\subsection{Spherical Angular Smoothing\label{sub:Spherical-Angular-Smoothing}}

\section{Results}

\subsection{Multi-fibre Simulations\label{sub:Multi-fiber-Simulations}}

\subsection{Software Phantoms\label{sub:Digital-Phantoms}}

\subsection{Results with software phantoms}

\subsection{Results with humans}

\section{Discussion and Conclusion}

\section*{Acknowledgements}
We gratefully acknowledge valuable discussions with Matthew Brett and Marta Correia on various aspects of this work.

\section*{Disclosure/Conflict-of-Interest Statement}
There are no conflicts of interest relating to this work.

\selectlanguage{british}%
\bibliographystyle{apalike2}
%\bibliographystyle{plainnat}
%\bibliographystyle{IEEEabrv, IEEEtran}
%\bibliographystyle{IEEEtran}
%\bibliographystyle{elsarticle-harv}
\selectlanguage{english}
\bibliography{diffusion}

\end{document}
