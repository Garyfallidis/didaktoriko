\documentclass{bioinfo}
\usepackage{url}

\usepackage[british,english]{babel}
\usepackage{mathpazo}
\usepackage[T1]{fontenc}
% \usepackage[latin9]{inputenc}
\usepackage{float}
\usepackage{bm}
\usepackage{nicefrac}
\usepackage{amsmath}
\usepackage{graphicx}
\usepackage{setspace}
\usepackage{amssymb}
\usepackage{natbib}
\usepackage[title]{appendix}
\usepackage{siunitx}
\usepackage{chngcntr}
\usepackage{algorithmic}
\renewcommand{\algorithmicrequire}{\textbf{Input:}}
\renewcommand{\algorithmicensure}{\textbf{Output:}}

\makeatletter
\newfloat{algorithm}{H}{loa}[section]
\floatname{algorithm}{Algorithm}
\counterwithout{algorithm}{algorithm}
\def\argmin{\mathop{\operator@font arg\,min}} 
\def\argmax{\mathop{\operator@font arg\,max}} 
\makeatother

\copyrightyear{}
\pubyear{}

\begin{document}
\firstpage{1}

\title[EIT]{Diffusion Imaging Q-Space Reconstructions on Cartesian Lattices}

\author[Garyfallidis, Yeh and
Nimmo-Smith]{Eleftherios~Garyfallidis\,$^{1,4,*}$, Frank~Yeh\,$^{2,*}$, Maxime~Descoteaux\,$^{3,*}$, and
  Ian~Nimmo-Smith\,$^{4}$\footnote{to whom correspondence should be
    addressed. e-mail: garyfallidis@gmail.com}}

\address{\,$^{1}$Wolfson College, University of Cambridge, Cambridge, UK\\
  \,$^{2}$Carnegie~Mellon, Pittsburgh.\\
  \,$^{3}$University of Sherbrooke, Canada.\\
  \,$^{4}$MRC Cognition and Brain Sciences Unit, Cambridge, UK.\\}

\history{}

\editor{}

\maketitle

\begin{abstract}
\noindent

<abstract to be written once the introduction is drafted>

\section{Keywords:} Diffusion Spectrum Imaging, Q-Ball Imaging,
Diffusion MRI.

\end{abstract}

\section{Introduction}

The fundamental principle in applying diffusion weighted MR (dMR)
imaging to the study of the neural fiber tracts in the brain is that the
directions of the tracts in each voxel are reflected in the way the
strength of the dMR signal varies with the magnitude and orientation of
each acquisition gradient.  \emph{Reconstruction} is the inverse process
by which we identify, or estimate, or infer) the fiber directions from a
voxel of diffusion weighted MR data. 

Some reconstruction methods can be used with data from a wide range of
acquisition designs; others by contrast require acquisition designs of a
prescribed type. Methods which are relatively independent of the q-space
structure of the acquisition include diffusion tensor imaging
\citep*[DTI;][]{Basser1994BiophysicalJ}. By contrast Q-ball imaging
\citep*[QBI;][]{Tuch2004} needs data to be sampled on one or more spherical
grids, and Generalized Q-sampling imaging \citep*[GQI;][]{Yeh2010},
requires the sampling to be on a cartesian grid.

Another important distinction between reconstrucion methods is whether
they are based on a (parametric) model or whether they are model-free
(nonparametric). Parametric methods include tensor, multitensor,
sticks-and-ball, and kurtosis models. These require fitting methods
which are typically iterative, apart from DTI. By contrast DSI, GQI and
spherical harmonic methods are model-free.

Non-parametric methods typically derive a function $\psi$ on a 2-sphere
$S^2$ representing all possible fiber orientations. Typically the larger
isolated maxima of $\psi$ are taken as local estimates of the directions of
underlying fibers.

We will use some qualitative characteristics describing the geometric
and physical properties of the dMR signal to identify a broad, new class
of reconstruction methods which require a cartesian grid q-space design.

\subsection{Notes to be turned into prose}

% Distinction between parametric (tensor, multitensor, sticks-and-ball,
% kurtosis) and non-parametric methods (DSI, GQI, spherical harmonic
% models). For non-parametric methods typically derive a function $\psi$
% on a 2-sphere $S^2$. This sphere represents all possible fiber
% orientations. Typically the larger isolated maxima of $\psi$ on $S^2$
% are taken as local estimates of directions of underlying fibers.

Fiber = real fiber, a packet of threads

Tract = fiber in the central nervous system (whose threads are axons)

ODF = orientation density function (not necessarily normalised). However
some ODFs are based on probabilistic assumptions where normalisation is
required - in this case ODF stands for orientation distribution function.

reconstruction sphere = spherical grid (mesh) whose vertices are the
locations where the ODF is evaluated.

estimated fiber directions = peaks (local maxima) of the ODF on the
reconstruction sphere

Goal: (1) a simple and general way to understand the diffusion
signal. (2) reliable resolution of low angle fiber crossings. (3)
consideration to performance (speed) of software implementations.
Principles:

we work directly on the measured signal in q-space -- dMR gradient
spaces -- (no Fourier transform required)

uses spatio-geometric properties of the signal - orthogonal information
+ additivity + decay

FRT works only on spherical shells - inner and outer evaluations of the
signal have something to contribute.

Introduce a framework $\int_r \int_{orth} F O$ which creates an ODF
where the equatorial radial projection of the contribution of signal is
based on $F$-transformed signal strength, and $O$-transformed radial
distance weighting

Limitation: data needs to be gathered on a suitably dense cartesian grid
in q-space - but can see how to extend to multiple shell acquisition
schemes.


\subsection{Overview}

\subsection{Theory}

\begin{methods}

\section{Methods and Materials}

\subsection{Equatorial Inversion Transform}

\subsection{Other methods}

\subsection{Implementation }

\subsubsection{Standard EIT\label{sub:Standard-EIT}}

\subsubsection{Fast EIT}

\subsection{Peak Finding\label{sub:Peak-Finding}}

\subsection{Spherical Angular Smoothing\label{sub:Spherical-Angular-Smoothing}}

\subsection{Data from software phantoms}

\subsection{Data from humans}

\end{methods}

\section{Results}

\subsection{Multi-fibre Simulations\label{sub:Multi-fiber-Simulations}}

\subsection{Software Phantoms\label{sub:Digital-Phantoms}}

\subsection{Results with software phantoms}

\subsection{Results with humans}

\section{Discussion and Conclusion}

\section*{Acknowledgements}
We gratefully acknowledge valuable discussions with Matthew Brett and Marta Correia on various aspects of this work.

\section*{Disclosure/Conflict-of-Interest Statement}
There are no conflicts of interest relating to this work.

\selectlanguage{british}%
\bibliographystyle{apalike2}
%\bibliographystyle{plainnat}
%\bibliographystyle{IEEEabrv, IEEEtran}
%\bibliographystyle{IEEEtran}
%\bibliographystyle{elsarticle-harv}
\selectlanguage{english}
\bibliography{diffusion}

\end{document}
