%% LyX 1.6.7 created this file.  For more info, see http://www.lyx.org/.
%% Do not edit unless you really know what you are doing.
\documentclass[twoside,english]{elsarticle}
\usepackage[T1]{fontenc}
\usepackage[latin9]{inputenc}
\pagestyle{headings}
\usepackage{url}
\usepackage{amsthm}

\makeatletter
%%%%%%%%%%%%%%%%%%%%%%%%%%%%%% Textclass specific LaTeX commands.
\theoremstyle{plain}
\newtheorem{thm}{Theorem}

%%%%%%%%%%%%%%%%%%%%%%%%%%%%%% User specified LaTeX commands.
% specify here the journal
\journal{Example: Nuclear Physics B}

\makeatother

\usepackage{babel}

\begin{document}



\title{This is a specimen title\tnoteref{t1,t2}}


\tnotetext[t1]{This document is a collaborative effort.}


\tnotetext[t2]{The second title footnote which is a longer longer than the first
one and with an intention to fill in up more than one line while formatting.}


\author[rvt]{C.V.~Radhakrishnan\fnref{fn1}\corref{cor1}\corref{cor2}}


\ead{cvr@river-valley.com}


\author[rvt,focal]{K.~Bazargan\fnref{fn2}}


\ead[url]{http://www.elsevier.com}


\fntext[fn1]{This is the specimen author footnote.}


\fntext[fn2]{Another author footnote, but a little more longer.}


\cortext[cor1]{Corresponding author}


\cortext[cor2]{Principal corresponding author}


\address[rvt]{River Valley Technologies, SJP Building, Cotton Hills, Trivandrum,
Kerala, India 695014}


\address[focal]{River Valley Technologies, 9, Browns Court, Kennford, Exeter, United
Kingdom}
\begin{abstract}
Abstract, should normally be not longer than 200 words.\end{abstract}
\begin{keyword}
quadruple exiton \sep polariton \sep WGM \PACS 71.35.-y \sep 71.35.Lk
\sep 71.36.+c \MSC[2008]23-557
\end{keyword}
\maketitle

\section{Introduction}

Bla bla, as listed in \citet{Parkin2005,LComp2004}.

\begin{eqnarray}
A & = & \prod_{i=1}^{\infty}B\nonumber \\
 & = & C\end{eqnarray}

\begin{thm}
Theorem 1
\end{thm}

\begin{thm}
Theorem 2
\end{thm}

\subsection{Subsection}
\begin{enumerate}
\item test
\item test
\end{enumerate}


\renewcommand{\labelenumi}{(\roman{enumi})} 
\begin{enumerate}
\item test
\item test
\end{enumerate}
\renewcommand{\labelenumi}{Step\,\alph{enumi})} 
\begin{enumerate}
\item test
\item test
\end{enumerate}

\subsubsection{Subsubsection}

Bla, bla

\appendix

\section{Appendix name}

Appendix, only when needed.


\section*{-----------------}

You can use either Bib\TeX{}:

\bibliographystyle{elsarticle-harv}
\addcontentsline{toc}{section}{\refname}\bibliography{../examples/biblioExample}



\section*{---------------------}

\noindent Or plain bibliography:
\begin{thebibliography}{2}
\bibitem{key-1}Frank Mittelbach and Michel Goossens: \emph{The \LaTeX{}
Companion Second Edition.} Addison-Wesley, 2004.

\bibitem{key-2}Scott Pakin. The comprehensive \LaTeX{} symbol list,
2005.
\end{thebibliography}

\end{document}
